\begin{multicols}{2}
    \Huge{Mynediad}\\
    \vspace{1cm}\\
    \normalsize{
      \noindent 
      (This is just a placeholder. A better description is needed.)\\

      Marches in the traditional music of this region served both practical 
      and social purposes. Historically, they accompanied processions, gatherings, 
      and even work, instilling a sense of order and collective movement.\\
      
      Musically, they are typically in 4/4 or 2/4 time, characterized by a steady, 
      even rhythm and a clear, strong pulse. Melodies are often straightforward 
      and memorable, designed for communal singing or instrumental performance. 
      Forms can vary, but often include repeated sections or variations.\\ 
      
      Culturally, marches fostered unity and could commemorate events or celebrate 
      figures within communities, forming an integral part of local traditions 
      and social life.

    }

    \newcolumn

    \Huge{Access}\\
    \vspace{1cm}\\
    \normalsize{
      \noindent
      (This is just a placeholder. A better description is needed.)\\
      
      Marches in the traditional music of this region served both practical 
      and social purposes. Historically, they accompanied processions, gatherings, 
      and even work, instilling a sense of order and collective movement.\\
      
      Musically, they are typically in 4/4 or 2/4 time, characterized by a steady, 
      even rhythm and a clear, strong pulse. Melodies are often straightforward 
      and memorable, designed for communal singing or instrumental performance. 
      Forms can vary, but often include repeated sections or variations.\\ 
      
      Culturally, marches fostered unity and could commemorate events or celebrate 
      figures within communities, forming an integral part of local traditions 
      and social life.
    }

    \end{multicols}