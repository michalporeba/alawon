\begin{multicols}{2}
    \Huge{Slides}\\
    \vspace{1cm}\\
    \normalsize{
      \noindent 
      (This is just a placeholder. A better description is needed.)\\

      Slides are lively dance tunes, primarily found in the southwestern regions. 
      They are characterized by their distinctive 12/8 time signature, which creates 
      a flowing, skipping rhythm with a strong emphasis on the first and fifth quavers 
      of each bar (often felt as two groups of six).\\
      
      This rhythmic structure gives the music a buoyant and slightly elongated feel 
      compared to jigs. Melodies are typically fast and often feature longer, 
      more lyrical phrases that complement the gliding quality of the dance steps 
      they accompany.\\ 
      
      While perhaps less universally common than jigs or reels across all areas, 
      slides hold a vibrant place in specific regional traditions, often played 
      at social gatherings and celebrations where their unique rhythm encourages 
      a particular style of movement.\\
      
      Their form commonly involves repeated sections (AABB), contributing to their 
      memorability and danceability.
    }

    \newcolumn

    \Huge{Slides}\\
    \vspace{1cm}\\
    \normalsize{
      \noindent
      (This is just a placeholder. A better description is needed.)\\

      Slides are lively dance tunes, primarily found in the southwestern regions. 
      They are characterized by their distinctive 12/8 time signature, which creates 
      a flowing, skipping rhythm with a strong emphasis on the first and fifth quavers 
      of each bar (often felt as two groups of six).\\
      
      This rhythmic structure gives the music a buoyant and slightly elongated feel 
      compared to jigs. Melodies are typically fast and often feature longer, 
      more lyrical phrases that complement the gliding quality of the dance steps 
      they accompany.\\ 
      
      While perhaps less universally common than jigs or reels across all areas, 
      slides hold a vibrant place in specific regional traditions, often played 
      at social gatherings and celebrations where their unique rhythm encourages 
      a particular style of movement.\\
      
      Their form commonly involves repeated sections (AABB), contributing to their 
      memorability and danceability.
    }

    \end{multicols}