\begin{multicols}{2}
    \Huge{Pibddawnsiau}\\
    \vspace{1cm}\\
    \normalsize{
      \noindent 
      (This is just a placeholder. A better description is needed.)\\

      Energetic social dances, likely emerging around the 16th century, 
      thrived in rural communities. Their music, typically in 4/4 time, 
      features a brisk tempo and a characteristic "spring" created by 
      dotted rhythms and syncopation. The tunes often follow a clear AABB structure, 
      with bright and singable melodies emphasizing a strong beat for dancing.\\

      These dances embody a vibrant community spirit, providing joyful entertainment 
      at gatherings. While rhythmically distinct, the music shares melodic traits 
      with other folk traditions of the region, existing alongside jigs and reels. 
      Similar lively dance forms can be found in related cultural areas.
    }

    \newcolumn

    \Huge{Hornpipes}\\
    \vspace{1cm}\\
    \normalsize{
      \noindent
      (This is just a placeholder. A better description is needed.)\\

      Energetic social dances, likely emerging around the 16th century, 
      thrived in rural communities. Their music, typically in 4/4 time, 
      features a brisk tempo and a characteristic "spring" created by 
      dotted rhythms and syncopation. The tunes often follow a clear AABB structure, 
      with bright and singable melodies emphasizing a strong beat for dancing.\\

      These dances embody a vibrant community spirit, providing joyful entertainment 
      at gatherings. While rhythmically distinct, the music shares melodic traits 
      with other folk traditions of the region, existing alongside jigs and reels. 
      Similar lively dance forms can be found in related cultural areas.
    }

    \end{multicols}